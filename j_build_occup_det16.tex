\documentclass[acmtodaes,notfinal]{acmtrans2m}

\usepackage{array}
\usepackage{multirow}
\usepackage{subfigure}
\makeatletter
\newcommand*{\rom}[1]{\expandafter\@slowromancap\romannumeral #1@}
\makeatother
\usepackage{tabu}
\newcolumntype{P}[1]{>{\centering\arraybackslash}p{#1}}

\usepackage{epstopdf}
\epstopdfsetup{update}
\usepackage{epstopdf-base}

\usepackage{graphicx}
\usepackage[cmex10]{amsmath}

\usepackage{booktabs}

\acmVolume{TBD}
\acmNumber{TBD}
\acmYear{TBD}
\acmMonth{TBD}

\let\orgsetcounter\setcounter
\usepackage{calc}

%\usepackage{cite}
%\usepackage{natbib}
\usepackage{graphicx}
\usepackage[dvips]{color}
\usepackage{epsfig}
%\usepackage{spacing}
\usepackage{subfigure}
\usepackage{amsmath}
\usepackage{amsfonts}
\usepackage{algorithm}
\usepackage{algorithmic}
\usepackage{url}
\usepackage{arydshln}
\usepackage{multirow}

\usepackage{lastpage}



\newcommand{\setRn}{\mathbb{R}_{}^N}
\newcommand{\setR}{\mathbb{R}}
\newcommand{\ud}{\,\mathrm{d}}
\newcommand{\res}[1]{\mathbf{r}^{(#1)}}
\newcommand{\vtr}[1]{\mathbf{#1}}
\newcommand{\mtx}[1]{\mathbf{#1}}
\newcommand{\mtxPhi}{\boldsymbol{\phi}}
\newcommand{\mT}{\mathrm{T}}

\newcommand{\real}{\mathrm{Re}}
\newcommand{\rad}{\mathrm{rad}}
\newcommand{\emin}{\mathrm{emin}}
\newcommand{\emax}{\mathrm{emax}}
\newcommand{\omin}{\mathrm{omin}}
\newcommand{\omax}{\mathrm{omax}}

\renewcommand{\algorithmicrequire}{\textbf{Input:}}
\renewcommand{\algorithmicensure}{\textbf{Output:}}
\renewcommand{\algorithmiccomment}[1]{\% \textit{#1}}

\newcommand{\BibTeX}{{\rm B\kern-.05em{\sc i\kern-.025em b}\kern-.08em
    T\kern-.1667em\lower.7ex\hbox{E}\kern-.125emX}}


\newcounter{numberlistc}
\newenvironment{numberlist}
    {   \setcounter{numberlistc}{0}
        \begin{list}{\arabic{numberlistc}.}
        {\usecounter{numberlistc}
        \setlength{\parsep}{0pt}
        \setlength{\topsep}{3pt}
        \setlength{\itemsep}{0pt}}
        }{ \end{list} }
\newcounter{itemlistc}
\newcounter{enumlistc}
%\renewcommand{\theromanlistc}{(\roman{romanlistc})} %for ref use
%\renewcommand{\thealphlistc}{(\alph{alphlistc})}    %for ref use
\newenvironment{itemlist}
    {   \setcounter{itemlistc}{0}
    \begin{list}{$\bullet$}
        {\usecounter{itemlistc}
        \setlength{\parsep}{0pt}
        \setlength{\topsep}{3pt}
        \setlength{\itemsep}{0pt}}
        }{ \end{list} }



% correct bad hyphenation here
\hyphenation{op-tical net-works semi-conduc-tor IEEEtran Gra-mian
  Gra-mians Kry-lov}


% paper title
\title{Thermal-Sensor-Based Occupancy Detection For Smart Buildings Using Machine Learning Methods}

\author{ \small
HENGYANG ZHAO\\
University of California, Riverside\\
QI HUA \\
Shanghai Jiao Tong University\\
HAIBAO CHEN and YAOYAO YE\\
Shanghai Jiao Tong University\\
HAI WANG \\
University of Electronic Science and Technology of China\\
SHELDON X.-D. TAN \\
University of California, Riverside\\
%XIN LI\\
%Duke University\\
ESTEBAN TLELO-CUAUTLE \\
INAOE and CINVESTAV, Mexico
}

\markboth{Hengyang Zhao \textit{et al.}}{Thermal-Sensor-Based Occupancy Detection For Smart Buildings Using Machine Learning Methods}

\begin{abstract}
  In this article, we propose a novel approach to detect the occupancy
  behavior of a building through the temperature and/or possible heat
  source information, which can be used for energy reduction and
  security monitoring for emerging smart buildings. Our work is based
  on a realistic building simulation program, EnergyPlus, from
  Department of Energy. EnergyPlus can model the various time-series
  inputs to a building such as ambient temperature, heating,
  ventilation, and air-conditioning (HVAC) inputs, power consumption
  of electronic equipment, lighting and number of occupants in a room
  sampled in each hour and produce resulting temperature traces of
  zones (rooms). Two machine learning based approaches for detecting
  human occupancy of a smart building are applied herein, namely:
  support vector regression (SVR) method and recurrent neural network
  (RNN) method. Experimental results with SVR method show that
  4-feature model provides accurate detection rate giving a 0.638
  average error and 0.0532 error ratio, and 5-feature model gives a
  0.317 average error and 0.0264 error ratio. This indicates that SVR
  is a viable option for occupancy detection. In RNN method, Elman's
  RNN (ELNN) can estimate occupancy information of each room of a
  building with high accuracy. It has local feedbacks in each layer
  and for a 5-zones building it is very accurate for occupancy
  behavior estimation. The error level, in terms of number of people
  can be as low as 0.0056 on average and 0.288 at maximum considering
  ambient, room temperatures and HVAC powers as detectable
  information. Without knowing HVAC powers, the estimation error can
  still be 0.044 on average, and only 0.71\% estimated points have
  errors greater than 0.5. Our study further shows both methods can
  deliver similar accuracy in the occupancy detection. But that SVR
  model is more stable for changing features of the system, while the
  RNN method can deliver more accuracy when the features used in the
  model do not change a lot.
\end{abstract}



\category{J.6}{Computer-Aided Engineering}{Computer-Aided Design}

\terms{Design, Algorithm}

\keywords{Smart building; support vector regression; neural network; indoor temperature; occupancy detection.}

\begin{document}

{\let\setcounter\orgsetcounter
\begin{bottomstuff}
\newline \indent
This work is supported in part by the Nature Science Foundation of China
(NSFC) under No. 61604095, in part by NSF Grant under No. CCF-1255899, in part by
Semiconductor Research Corporation (SRC) grant under No. 2013-TJ-2417, and in part by UC MEXUS-CONACYT
Collaborative Research Grants CN16-161.  

%This work is supported by the National Science Foundation, under
%  grant CCF-1255899 and grant CCF-1527324 and UC MEXUS-CONACYT
%  Collaborative Research Grants CN16-161.

\end{bottomstuff}
}

\maketitle

\input intro.tex

\input energyplus_review.tex

\input machine_learning_review.tex

\input proposed_approach.tex

\input experiments.tex

%\input svm_nn_method_results.tex
%
%\input rnn_method.tex
%%
%\input results.tex

\section{Conclusion and future works}
\label{sec:conclution}

In this article, we propose a machine learning based method to
detect the occupancy behavior of a building through the temperature
and/or possible heat source information. Supporting vector regression and recurrent neutral
network methods are developed for smart buildings through the thermal sensor
temperature information and/or possible heat source information, have
been discussed. In all experiments, we used the realistic building
simulation program EnergyPlus to collect training and validation
data sets. Ambient factors, room temperature, and/or HVAC power were
selected as features to train Elman's recurrent neural network.
In SVR model, two sets of features are offered to feed off the
model for different conveniences. The first set of features is
comprised of 4 features including solar factor, working time, indoor
temperature and outdoor temperature, which are regarded as easily
obtained features; whereas the second set of features adds light
energy as the fifth feature. In light of the experimental results,
4-feature model has a quite accurate detection rate which gives a
0.638 average error and 0.0532 error ratio. However, 5-feature SVR
model giving a 0.317 average error and 0.0264 error ratio has a better
performance than 4-feature model, which we consider as moderating the
under-fitting issue. This indicates that using SVR model is a viable
option when it comes to occupancy detection given its convenience in
data acquirement. In the recurrent neutral network based method, the
resulting Elman network can estimate occupancy information of each
room of a building with high accuracy. Using ambient factors and room
temperatures only, the average estimation error is 0.044, and only
0.71\% of the estimated points have errors greater than 0.5 in terms
of number of people. This indicates that it is possible to precisely
estimate the occupancy only using ambient factors and room
temperatures. With HVAC powers added, the estimation can be even more
accurate with even simpler neural networks.  Our study further shows
both methods can deliver similar accuracy in the occupancy detection.
But that SVR model is more stable for changing features of the system,
while the RNN method can deliver more accuracy when the features used
in the model do not change a lot.

% As a conclusion, what influences the model is the number of features
% and the number of data points, virtually the two different methods SVR
% and RNN work similarly well for solving this problem. However,
% further improvement is likely to occur when the model is refined to be
% able self-improved using current data set in the future.



% The proposed method has been tested and validated on a 3D
% IC structure.  Numerical results on a 3D IC circuit also show the
% clear advantage of the proposed method over the LU-based CSPARSE
% solver in terms of scalabilities.

% \section{Acknowledgments}
% \label{sec:acknowledgements}

% The authors would like to thank Prof. Zhou Dian from University of
% Texas, Dallas for the valuable introduction and discussion on the
% Kharitonov's polynomial theorem and other analog analysis topics.

\bibliographystyle{acmtrans}
%\bibliography{../../bib/mscad_pub,../../bib/thermal_power,../../bib/stochastic,../../bib/architecture,../../bib/simulation,../../bib/modeling,../../bib/reduction,../../bib/interconnect,../../bib/misc}
%\bibliography{../../bib/hmatrix_ref,../../bib/reliability.bib,../../bib/reliability_papers,../../bib/stochastic,../../bib/simulation,../../bib/modeling,../../bib/reduction,../../bib/misc,../../bib/architecture,../../bib/mscad_pub,../../bib/thermal_power}
\bibliography{title2}

% \section*{Submission Note}
% Some preliminary results of this paper appeared in \emph{2014
% International Symposium on Quality Electronic Design (ISQED 2014)}
% \cite{Yu:2014}. This submission has the following changes compared
% to our conference version:

% \begin{itemlist}
% \item The content of the paper, including the notations and the
% figures, has been substantially revised to improve the presentation.
% The abstract and the introduction have been also rewritten to
% reflect the new scope of the paper.

% \item The proofs of the $\mathcal{H}$-matrix representations for the
% thermal matrix and its inverse have been given, and some new results
% on the rank in the $\mathcal{H}$-matrix representations have been
% obtained. We also had some further discussions on the adaptive rank
% and the computational complexity.

% \item For the simulation of the 3D-IC model, the comparison of our
% proposed method with the known CSPARSE method is made. Some more
% simulation results have been given to show the effectiveness of our
% method over the LU-based CSPARSE method.

% \item Added comparison results and discussions for another two
%   solvers, UMFPACK and CHOLMOD.

%\end{itemlist}

\input submission_note.tex


%\input replyNov.tex
%
% \begin{received}
% %Received Month Year; revised Month Year; accepted Month Year
% \end{received}

\end{document}

%%%%%%%%%%%%%%%%%%%%%%%%%%%%%
%%%%%% File Stops Here %%%%%%
%%%%%%%%%%%%%%%%%%%%%%%%%%%%%

% Hao's significant spelling errors, such as ``s-expended'',
% are corrected.
%
% Manually enforced line spacings are appropriately adjusted.
%
% Overlong tables are modified.
%
% Bad fonts and shapes in texts and equations are replaced.
