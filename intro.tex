\section{Introduction}
\label{sec:intro}
Building takes an instrumental role in energy consumption and smartness of a
building has a large impact on inhabitants. \textcolor{feb18rev}{According to
statistics provided by US Department of Energy, more than 70\% of electricity
has been consumed by buildings every year \cite{doe_annual}. Recent efforts
have been poured into the awareness of
improving efficiency in quite a few facets, e.g., heating, ventilation, air
conditioning (HVAC) system \cite{erickson2009energy}\cite{gao2009selfprog},
lighting\cite{delaney2009eval}, IT (Information Technology) energy consumption management within
buildings\cite{agarwal2009augnet}\cite{agarwal2010sleep}, etc.} Amongst the
overall energy usage of various aspects of buildings, the efficiency of HVAC
systems has a tremendous impact on energy consumption \cite{hobby2012analysis}.
\textcolor{feb18rev}{On the contrary, some buildings utilizing scheduled-based
programmable thermostats may consume more energy than ones without using smart
devices, when acutal occupancy deviates from the programmed schedule}
\cite{bias1999elec}. Automatic thermostat control systems have been developed
in different approaches \cite{thomas2012intelligent}\cite{lu2012eval}, and
plenty of techniques are applied in the course of building the system.

Detecting the occupancy (i.e. whether there are residents) in a building or a
room has applications ranging from energy reduction to security monitoring. For
instance, occupancy detection is critical for energy and comfort management
system in a smart building~\cite{Nguyen2013Energy}.  Using the occupancy
information, HVAC and lighting can be automatically controlled to reduce energy
consumption while keeping human comfort. Energy consumption optimization can
achieve higher quality with knowledge of current building occupancy
\cite{majumdar2014energy}. Previous research shows that energy
can be saved by 28\% by automatically sensing occupancy and turning off HVAC
when the building is not occupied~\cite{Lu:SenSys10}.

Due to the importance of detecting building occupancy, many methods have been
proposed in the past. The most widely used method is by means of motion
detection (using different techniques such infrared, RF, sounds, vibrations
magnetism etc), which can detect if there is person or not. But motion
detection in general can not tell how many persons in a room.  Other methods
include passive infrared sensors~\cite{Dodier2006Building}, wireless camera
sensor network \cite{erickson2009energy}, and applying sound level, case
temperature, carbon-dioxide ($CO_2$) and motion to estimate occupancy
number~\cite{Ekwevugbe2013Real}.  Preheat \cite{scott2011ctrl} built rooms with
active radio frequency identification (RFID) and sensors to detect home
occupancy. Mozer \cite{mozer1997neurothermostat} proposed a neural network
method by using the history data from embedded motion sensors and actives RFID
to explore occupancy rate. Thermostat \cite{lu2010smartthermostat} also devoted
a similar approach through the employment of magnetic reed switches and passive
infrared sensors to take control of the HVAC system at home. However, those
methods are more expensive for deployment as dedicated equipment is required.
\textcolor{feb18rev}{For the work \cite{dong2014real}, it uses $CO_2$, motion, and acoustics to detect occupancy and achieves 92\% accuracy.
The work \cite{majumdar2014energy} uses $CO_2$ and motion sensors to build a probability model using 3-month data, which is used in energy-comfort optimization. A multiple sensor network method is proposed in \cite{yang2016review} for occupancy detection. Different from these methods, we will build a machine learning model for occupancy detection by carefully selecting some features including solar angle, indoor temperatures, outdoor temperatures, working time, and lights energy. Comparing to the work \cite{dong2014real}, our RNN method can achieve higher than 99\% accuracy using two or more hidden recurrent layers. Artificial neural network (ANN) is applied in occupation detection works
\cite{lam2009occupancy} in a feed-forward manner, not capturing periodical
patterns as using a recurrent neural network. Other machine learning
techniques, such as support vector machine (SVM) and hidden Markov model are
also evaluated and 75\% estimation accuracy was achieved. The above works didn't
take historical data into account, which is important in patterned (e.g.
weekly, seasonal) occupancy prediction.}

One viable and cost-efficient  approach for the occupancy detection is to leverage
existing temperature sensors or temperature sensor networks already
deployed in many residual and commercial buildings. As human being
will lead to small disruptions or perturbations of temperature in a
room, temperature sensor information can be analyzed to detect the
occupancy and even the number of persons without additional costs.
% Many works have given an approach under the circumstance that the
% detection requires a comparatively strict requirements for sensors,
% and obviously the requirements of sensors resulting in transformations
% of infrastructures may dramatically increase expenditure when it comes
% to the total cost of the building and system. Besides, overflow data
% including a vast of different aspects of conditions with respect to
% inhabitant data looms a latent possibility to burden inhabitants
% psychological pressure, because a large number of people may not
% prefer living under the supervision of a great deal of data that is
% available to someone else. Moreover, it may lead to a threat of
% leakage of personal data, thus threatening personal privacy.
% The approach we propose is capable of detecting the occupancy of
% employees given conditions of specific parameters of an office,
% including the building structure, wall material, infiltration
% coefficients, etc. We set up a vivid and real simulation of building
% in EnergyPlus to acquire the data set under certain conditions. In
% this model, sensors are not required to be highly precise or extremely
% accurate which are demanded in a number of previous models proposed by
% other approaches, and the model we proposed does not detect employee
% occupancy through precise detection of state-of-the-art sensors. The
% occupancy of employees is detected under statistic approach and
% historical data set, the result of which will not rapidly fluctuate
% when subtle ambient temperature changes occur.
% EnergyPlus is a whole building energy simulation program
% \cite{energyplus} that utilized by researchers, architects and
% engineers to model energy consumption, e.g., ventilation, lighting,
% heating, cooling, and plug and process loads, and water use in
% buildings with U.S. Department of Energy Building Technologies Office
% funding its development. EnergyPlus plays in the role as a modular and
% structured code which contains most popular features. It mainly is a
% functional engine with texts written in format of inputs and outputs.
% Via a heat balance engine, loads calculated is able to be modified by
% users in specific time step to simulate the building system. The
% EnergyPlus building systems simulation module which is with the
% variable time step is capable of calculating cooling and heating
% system and the response of electrical system. This system works more
% like an integrated simulation and it is playing an important role in
% calculations for plant sizing, system, occupant health calculations
% and occupant comfort while providing precise temperature detection.
% Moreover, integrated simulation also offers user authority to evaluate
% moisture adsorption and desorption, controls of realistic system in
% building elements, air flow in an interzone, and radiant cooling and
% heating systems. Data generated from EnergyPlus is most likely to be
% regarded as the authentic conditions of an office. In this model, a
% segment of data set is acquired from EnergyPlus, with which we combine
% real data to make detection on office occupancy.
% Due to the fact our approach is based on the existing office located
% in Chicago, hence it is highly related to the infrastructures of the
% office upon which we build our model including internal materials,
% internal loads, space conditioning, location, simulation period,
% ground temperatures, walls, floors, roofs, etc. Before we establish
% this model, we need to confirm some parameters of the office of a
% building. This model has a high demand for knowing the detail of a
% certain room or office, even the infrastructure of a building. We need
% the specific parameters of a room to establish a model and get access
% to a vast amount of data generated via EnergyPlus.
In this work, we propose a novel approach to detect occupancy based on
the temperature sensor information under specific conditions by
applying machine learning methods, while it does not require plenty of
sensors to be installed in a certain building.  We generate the
mathematical model based on support vector regression (SVR) and
recurrent neural network (RNN) to detect occupancy with two sets of
features for different application situations. \textcolor{feb18rev}{We
  remark that SVR and RNN methods are widely used in non-linear
  regression applications because of their good interpolation
  performance.  We started with SVR as it is a widely used traditional
  machine learning algorithm. Then we tried the RNN network for the
  same problem as an RNN is able to efficiently capture
  frequency-domain characteristics in its recurrent neurons, modeling
  a non-linear dynamic system underlying, which is an advantage of
  being awareness of daily, weekly, and seasonal occupation patterns.
  As a result, we can compare the traditional machine learning
  algorithm with recent deep learning technique for our problem. The
  comparison indeed sheds some interesting lights on this problem for
  the two methods as discussed later.  It should be also noted that
  other machine learning method such as the traditional back
  propagation (BP) neural network algorithm can be used for occupancy
  detection for smart buildings.}

\textcolor{feb18rev}{The new approach is based on machine learning
  approaches in which thermal-related features including room
  temperature, ambient temperature, and other related heat sources for
  detecting the number of people in a room.  The experiments are
  performed by using a realistic building simulation program,
  EnergyPlus \cite{energyplus:2001vf}, from Department of Energy,
  which can model the various time-series inputs to a building such as
  ambient temperature, heating, ventilation, and air-conditioning
  inputs, power consumption of electronic equipment, lighting and
  number of occupants in a room sampled in each hour and produce
  resulting temperature traces of zones (rooms).}  Experimental
results with SVR method show that 4-feature model provides accurate
detection rate giving a 0.638 average error and 5.32\% error rate, and
5-feature model gives a 0.317 average error and 2.64\% error
rate. This indicates that SVR is a viable option for occupancy
detection.  In RNN method, we apply the Elman's recurrent neural
network, which has local feedbacks in each layer. We use a simple
formula to calculate the RNN layer number, layer size to configure RNN
architecture to avoid overfitting and underfitting problems. The error
level, in terms of number of people can be as low as 0.0056 on average
and 0.288 at maximum considering ambient, room temperatures and HVAC
powers as detectable information. Without knowing HVAC powers, the
estimation error can still be 0.044 on average, and only 0.71\%
estimated points have errors greater than 0.5. Our study further shows
both methods can deliver similar accuracy in the occupancy
detection. \textcolor{feb18rev}{But the SVR model is more stable for
  adding or removing features of the system, while the RNN method can
  deliver more accuracy when the features used in the model do not
  change a lot. Comparing to first-principle based methods, SVR and
  RNN methods work in a black-box way, which reduces the efforts of
  explicitly building model of occupancy detection and sensor network
  deployment, while good detection accuracy can still be achieved, as
  shown in the result section.}\textcolor{feb18rev}{Our major
  contributions are listed as following:
\begin{itemize}
\item We use thermal system consistency to detect occupancy using other
    thermal-related parameters acquired by existing sensors, which dramatically
    reduces the deployment effort, such as elimination of the requirement
    of infrared sensors.
\item We apply machine learning techniques, such as deep recurrent neural
    networks, to explore the possibility and accuracy of occupancy detection.
    This does not require explicitly application-specific modeling.
\item Two machine learning methods are thoroughly compared in terms of accuracy,
    using different feature set, which highlights the importance of certain
    significant features, such as HVAC power consumption.
\item We use dynamic machine learning models (e.g. recurrent neural network
    instead of non-recurrent) such that periodical occupancy patterns can be taken
    into account.
\end{itemize}
}

The rest of this paper is organized as the following.
Section~\ref{sec:energy_plus_review} reviews the EnergyPlus program used for
generating realistic building data. Section~\ref{sec:machine_learning_review}
reviews the two machine learning methods SVR and Elman's recurrent neural
network (ELNN) used in this work. Section~\ref{sec:proposed_methods} introduces
the two methods for the given problems of occupancy detections based on the
thermal sensor information. Then the experimental results based on the two
methods, discussions and comparison between the two methods are presented in
Section~\ref{sec:results}. Section~\ref{sec:conclution} concludes this paper.
