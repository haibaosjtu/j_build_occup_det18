\newpage
\onecolumn

\setcounter{page}{1} \pagenumbering{roman}\vskip 2pc

\centerline{\LARGE \bf Summary of Revision}
\centerline{\Large (paper id: No.~TODAES-2017-P-1537)}

\section{Revision summary}

First, we would like to thank the Editor in Chief and associate editor
and the four referees for their careful reviews of our manuscript,
suggestions and comments that help further improve the presentation of
our work.  A revision of the paper has been completed, all the review
comments have been addressed and we hope the revised paper will
resolve all the concerns of the reviewers.  All the major changes are
highlighted in \textcolor{red}{RED} in this revision.

Before we reply the questions of each reviewer, we would like to
highlight the following major modifications:

\begin{itemlist}

\item The main concern of the manuscript has been addressed in the
introduction section and more citations have been added.

\item Some figures have been updated and the more
discussions about comparisons with existing works have been added in
the introduction section and the numerical experiment section.

\item All the mentioned presentation issues have been addressed.

\end{itemlist}

Below, we will answer the questions raised by each reviewer. For the
completeness, the original comments are {\it quoted} first, followed
by a short explanation of how the comments have been addressed.


\vspace{0.2in}


\section{Answer to reviewers' comments}

\subsection{Answers to Reviewer 1's comments}
\begin{verbatim}
(1) This paper presents a approach to detect the occupancy behavior
of a building through the temperature and/or possible heat source
information. The building simulation software EnergyPlus is used
to generate temperature traces of rooms under given various
time-series, such as ambient temperature, and HVAC inputs.
And SVR and RNN are used to detect human occupancy of a smart
building, respectively. However, this paper has limited innovation
in both technique and results. There are two main comments or
concerns from the reviewer. Why RNN and SVR can be used to detect
the occupancy behavior in this paper? In other words, why not use
other machine learning methods and other optimal methods?
Authors should give further explanations in terms of theory and
experiments.
\end{verbatim}

Answer: Thanks for your comments. SVR and RNN methods are widely used
in non-linear regression applications because of their good
interpolation performance. Therefore, RNN and SVR can be used to
detect the occupancy behavior. Especially, an RNN is able to
efficiently capture frequency-domain characteristics in its recurrent
neurons, modeling a non-linear dynamic system underlying, which is an
advantage of being awareness of daily, weekly, and seasonal occupation
patterns. In this paper we want to address the comparison between
traditional machine learning algorithm with deep learning
technique. SVR is a traditional machine learning algorithm which is
widely used as well. It should be also noted that other machine
learning method such as the traditional back propagation (BP) neural
network algorithm can be used for occupancy detection for smart
buildings. Actually we did the experiments by using the BP neural
network and found that this method can also be applied for detecting
human occupancy (details are omitted because of limited space). 

We have addressed this in the introduction section of the revised
manuscript.

\begin{verbatim}
(2) Authors should give some comparisons with other works in
experiments, such as:

[1] Dong, Bing, and Khee Poh Lam. "A real-time model
predictive control for building heating and cooling systems based
on the occupancy behavior pattern detection and local weather
forecasting." Building Simulation. Vol. 7. No. 1. Springer
Berlin Heidelberg, 2014.

[2] Majumdar, Abhinandan, et al. "Energy-comfort optimization
using discomfort history and probabilistic occupancy prediction."
Green Computing Conference (IGCC), 2014 International. IEEE, 2014.

[3] Yang, Junjing, Mattheos Santamouris, and Siew Eang Lee. "Review
of occupancy sensing systems and occupancy modeling methodologies
for the application in institutional buildings." Energy and
Buildings 121 (2016): 344-349.

I cannot list all works about occupancy detection. Authors should
give sufficient survey for related works.
\end{verbatim}

Answer:
Thanks for pointing this out. Some more existing works have been reviewed and some more citations of related works have been added, please see the introduction section of the revised revision. 

For the work [1] mentioned by the reviewer, it uses CO$_2$, motion, and acoustics to detect occupancy and achieves 92\% accuracy. The work [2] uses CO$_2$ and motion sensors to build a probability model using 3-month data, which was used in energy-comfort optimization. The authors of [2] didn't provide occupancy detection precision. The work [3] have one major disadvantage (mentioned in [3]/Table2): multiple sensor network needed. Different from these methods, we build a machine learning model for occupancy detection by carefully selecting some features including solar angle, indoor temperatures, outdoor temperatures, working time, and lights energy. Comparing to the work [1], which provided accuracy data (92\%), our RNN method can achieve higher than 99\% accuracy using two or more hidden recurrent layers.

\begin{verbatim}
(3) In addition, there are other several comments or concerns about
this paper. In page 2, 21 line, what is "IT"?
\end{verbatim}
Answer: IT means information technology. In the paper it means energy
management system with information technology. We have added the explanation in
Section \ref{sec:intro} of the revised manuscript.

\begin{verbatim}
(4) In page 2, 24 line, why "buildings utilizing programmable
thermostats virtually are more likely to consume more energies
than ones without using smart devices"? Reviewer cannot find
the same opinion in Ref [Bias and Cheng1999]".
\end{verbatim}
Answer: Buildings utilizing programmable
thermostats virtually are more likely to consume more energies than ones
without using smart devices because the smart devices often set off wrongly
and causes unnecessary cost.

\begin{verbatim}
(5) Typo in page 3, 50 line, provide -> provides. Please
carefully proofread the whole paper.

What is x in page 5, 45 line; What is l in page 5, 54 line;
What is y in page 6, 9 line. Please give definitions for
all variables.

Syntax error in page 6, line 52-53: "We apply the Elman
recurrent neural network architecture as shown in Fig. 4 to
the occupancy of each room in a certain smart building.".
Please carefully proofread the whole paper.
\end{verbatim}
Answer: We have proofread the whole paper and the typos have been corrected in
the revised manuscript. The definitions for the variables have been added in
Section \ref{sec:machine_learning_review} of the revised revision. All the mentioned presentation issues have
been addressed in the revised manuscript.

\begin{verbatim}
(6) In Tab I and Tab II, authors use absolute error to
evaluate estimation performance. However, authors should
give real-value of occupancy. Why not use relative error
to evaluate that? It seems be more reasonable.
\end{verbatim}
Answer: The real-values of occupancy (people count in room) are shown in Figures
\ref{fig:compare1}, \ref{fig:compare2}, \ref{fig:compare3},
\ref{fig:one-layer}, and \ref{fig:two-layer}. Y-axis represents number of
people. We use relative error (error rate) to evaluate occupancy detection, as
shown in Tables \ref{table:training}, \ref{table:testing}, \ref{tab:terr-stat},
and \ref{tab:verr-stat}. We slightly changed the table formats so that they use
unified error rate to deliver better comparison, as visualized in Figure
\ref{fig:accuracy-comparison}.

\begin{verbatim}
(7) How to determine constant C in SVR.
\end{verbatim}
Answer: Constant C in SVR is one of its hyper parameters. In this paper we used
grid search to determine the value C. We added the explanation in Section ????.

\subsection{Answers to reviewer \#2's comments}
\begin{verbatim}
(1) This paper uses SVR and RNN to predict the occupation
(i.e., the number of people in a room). The proposed technique
can be provided to smart buildings to optimally control
the HVAC system. A smart building can shut down the HVAC
system for empty rooms. It can save 28\% electricity energy
according to the previous research. The proposed method is
demonstrated based on the simulation results by EnergyPlus.
This reviewer would like to bring up the following comments.

Comparing the SVR and RNN models, different features are used.
For SVR, the features include solar angle, light energy,
outdoor temperature, indoor temperature and working time.
For RNN, other ambient factors and HVAC power are used. Why
do the authors use different features here? Ideally, the same
features should be used for different models in order to make
a fare comparison.
\end{verbatim}

Answer:
Thanks for your comments. Five main features are applied in the machine learning methods for occupancy detection.
These features covers all major thermal-related aspects to be coupled with the human activity, such that
occupancy can be determined theoretically. In order to investigate the impact
of choosing different set of features and to improve the flexibility of practical
deployment, different combination of features are used in the methods. We have addressed this in the feature selection Section
\ref{sec:proposed_methods}.

For a fare comparison between SVR and RNN, we do use the same feature
set for a fare comparison. We have clarified this point in the comparison Section \ref{sec:comparison_svr_rnn} of the revised
revision.

\begin{verbatim}
(2) The authors claim that the SVR model is more stable if
the features are changed. The definition of ``changing
features'' is not clear. Please clarify.
\end{verbatim}

Answer: Changing features means to add or remove one feature in the feature pool. SVR
performs a little more stably from the experiment results. We have clarified
this in the abstract, the conclusion, and the introduction Section \ref{sec:intro} of the revised manuscript.

\begin{verbatim}
(3) The authors discuss the accuracy and characteristics based on
Figure 16. However, it is very hard to observe any difference
between the curves generated from SVR, RNN and EnergyPlus.
\end{verbatim}

Answer: Indeed, both SVR and RNN methods can deliver similar accuracy in the occupancy detection with EnergyPlus,
as shown in Figure \ref{fig:comparison}. To remedy this, we added a new figure to further show the differences in accuracy between the proposed SVR and RNN models and the method in \cite{dong2014real} in Section \ref{sec:results} of the revised version. 

\begin{verbatim}
(4) The training and testing data are generated from EnergyPlus
by simulation other than physical measurement. The accuracy of
EnergyPlus simulation cannot be guaranteed. Please discuss
the pros and cons of such a simulation-based approach.
\end{verbatim}

Answer: EnergyPlus has been proven as a successful energy simulation program with
widely-accepted accuracy for modern buildings (two citations added to the
paper).  Instead of conducting real building experiments, using EnergyPlus to
test and simulate the proposed models is therefore valid, with a state of art
simulation accuracy, which was a major pain point in the past. On the other
hand, using EnergyPlus allows the authors to efficiently choose from a variety
of buildings to test on, as well as easily configure the energy inputs.

We have added some discussions about the pros and cons of such a simulation-based
approach in Section \ref{sec:energy_plus_review} of the revised manuscript.

\begin{verbatim}
(5) There are several other minor issues. In the second paragraph of
Introduction, the authors mention that 70% of the energy has been
consumed by buildings according to the US department of energy.
A citation is needed. In addition, a citation for EnergyPlus
is needed as well.
\end{verbatim}

Answer: A Dept Energy citation was added to support the 70\% consumption
percentage.  URL (doesn't appear in the paper due to the formatting):
https://www.eia.gov/totalenergy/data/annual/index.php\#consumption An
EnergyPlus citation was added for the EnergyPlus software. We have added the
citation entries in Section \ref{sec:energy_plus_review} of the revised version.

\begin{verbatim}
(6) There is no label for the x-axis in Figures 10, 11, 12, 14 and 15.
\end{verbatim}

Answer: Fixed. In the captions, we added descriptions for x-axis range, unit, and
sample period.

\subsection{Answers to reviewer \#3's comments}
%\vspace{0.2in}\emph{Review number 3.}
\begin{verbatim}
(1) Comments to the Author: In this model, we provide two sets of
model which offers different extent of convenience to detect
number of employees in an office. I have no clue what the above
sentence is trying to say. There are many more like this. Occupancy
determination is highly studied in the literature. Authors don't seem
to be aware of the variety of work done in the area. I don't know
what the paper is about and what its novelty is.
\end{verbatim}

Answer: Thanks for your comments. We have revised this sentence in Section \ref{sec:proposed_methods} of
the revised manuscript. We have proofread the whole paper and the presentation issues have
been addressed in the revised version. The main concern of the manuscript has been addressed in the
introduction section and more citations have been added. Also, some more discussions about comparisons
with existing works have been added in the introduction section and the numerical experiment section.

Our work focuses on applying machine learning techniques, such as deep recurrent neural networks, to
explore the possibility and accuracy of occupancy detection. This does not require explicitly application-specific modeling.
Two machine learning methods are thoroughly compared in terms of accuracy, using different feature set, which highlights 
the importance of certain significant features, such as HVAC power consumption. We have added a list at the end of
Section \ref{sec:intro} to highlight our contribution and novelty.

\subsection{Answers to reviewer \#4's comments}
\begin{verbatim}
(1) Comments to the Author: A machine learning based method is
proposed in this paper to detect the occupancy behavior of
a building. The paper is well-organized and well-written.
The rationale behind the method is reasonable and interesting.
I list my main concerns as below.

Related works are not completely reviewed. More related works
should be introduced to draw a big picture for the readers.
\end{verbatim}

Answer: Thanks for your support. Some more existing works have been reviewed
and some more citations of related works have been added in the introduction
Section \ref{sec:intro} of the revised manuscript.

%We extended a graph in Section \ref{sec:intro} to briefly describe
%ANN (artificial nerual network) application in this area. We added one more
%citation in this area. ANN, SVM and hidden Markov models were involved.


\begin{verbatim}
(2) Extensive simulations have been carried out to validate the
effectiveness of the proposed method, and many experimental
results are shown in the paper. However, few result analyses
are given in the paper to explain the reason why the proposed
approach can achieve better performance.
\end{verbatim}

Answer: Comparing to the referencing result, the occupation detection performance is
satisfactory, in terms of the detection accuracy. It is superior than others
in the following aspects: 1) No explicit requirement of body-detection
devices, such as infrared detector, which reduces the deployment effort and
cost; 2) Machine learning methods are black-boxed which do not require
explicit modeling.

Introduction section was revised to reflect the points mentioned above.

\begin{verbatim}
(3) How to select the optimal parameters of SVR and RNN? It is
well known that the performance of SVR and RNN heavily depends
on parameter selection.
\end{verbatim}

Answer:
We revised section ``feature selection'' under ``proposed occupancy estimation
approaches''. All the five thermal-related features should be used in order to
achieve the highest detection accuracy. However, it is sometimes not practical
to acquire all the information, such as the lighting energy, as described.  In
such situation, we can observe a detection accuracy degradation, as described
in the experiment result section.



\begin{verbatim}
(4) The authors should clearly point out the major contributions
of this paper by using 3 to 5 brief bullet points.
\end{verbatim}

Answer: We have added a list at the end of introduction section.


\begin{verbatim}
(5) You need to decrease the number of pages since there are some
unnecessary explanations especially in the experimental parts.
Suppose the readers are familiar with basic concepts. Overall,
I think the extension has enough material to be considered for
publication in TODAES.
\end{verbatim}

Answer: We have deleted redundant explanations in the revised manuscript.
